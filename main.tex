\documentclass[titlepage,11pt]{article}
\usepackage{enumitem}
\usepackage{listings}
\usepackage{amsmath}
\usepackage{graphicx}
\usepackage[font=small,labelfont=bf]{caption}
\usepackage[bahasa]{babel}
\usepackage{float}
\usepackage{verbatim}
\usepackage{graphicx,tabularx,multirow}
\usepackage{xcolor}
\usepackage[onehalfspacing]{setspace}
\usepackage[
	allcolors=visigrey,
	colorlinks=true,
]{hyperref}
\usepackage[a4paper,left=2cm,right=2cm]{geometry}

%Code listing style pak akok
\definecolor{codegreen}{rgb}{0,0.6,0}
\definecolor{codegray}{rgb}{0.5,0.5,0.5}
\definecolor{codepurple}{rgb}{0.58,0,0.82}
\definecolor{backcolour}{rgb}{0.95,0.95,0.92}

\lstdefinestyle{mystyle}{
	backgroundcolor=\color{backcolour}, commentstyle=\color{codegreen},
	keywordstyle=\color{magenta},
	numberstyle=\small\color{codegray},
	stringstyle=\color{codepurple},
	basicstyle=\ttfamily\footnotesize,
	breakatwhitespace=false,         
	breaklines=true,                 
	captionpos=t,                    
	keepspaces=true,                 
	numbers=left,                    
	numbersep=5pt,                  
	showspaces=false,                
	showstringspaces=false,
	showtabs=false,           
	frame = single,
	tabsize=2
}
\lstset{style=mystyle}

\definecolor{visigrey}{rgb}{.1,.15,.15}
\geometry{top=1cm,bottom=.5cm}
\savegeometry{titlepage}
\geometry{top=2cm,bottom=2cm}
\savegeometry{main}

\def\bspace{\(\qquad\qquad\qquad\)}
\usepackage[T1]{fontenc}
\usepackage[utf8]{inputenc}
\usepackage{tgheros}
\renewcommand*\familydefault{\sfdefault}

\setcounter{tocdepth}{6}

\def\autor{Laboratorium }
\def\lab{Multimedia dan Internet of Things}
\def\departemen{Departemen Teknik Komputer}
\def\institut{Institut Teknologi Sepuluh Nopember}
\def\praktikum{Praktikum \\ Dasar Pemrograman}
% Ubah Judul sesuai dengan modul
\def\judul{Fungsi dan Rekursi}
\def\tahun{2023}
\begin{document}
% Ubah Bahasa sesuai dengan keinginan
\selectlanguage{bahasa}
\input{Cover/Header.tex}
% Pilih Modul yang akan di build
% \section{Tujuan}
\begin{itemize}[label=$\bullet$, itemsep=-1pt, leftmargin=*]
	%    \setlength\itemsep{0.5em}
	% \item Students are able to create projects on an IDE
	% \item Students can demonstrate his/her knowledge of the structure of a C program
	% \item Students can demonstrate his/her knowledge of C data types
	% \item Students can demonstrate his/her knowledge of C operators
	% \item Students are able to use function to read inputs from keyboard
	% \item Students are able to use function to print texts on screen
	\item Mahasiswa dapat membuat proyek di dalam IDE
	\item Mahasiswa dapat menunjukkan pengetahuan mereka tentang struktur program dalam bahasa C
	\item Mahasiswa dapat menunjukkan pengetahuan mereka tentang tipe data dalam bahasa C
	\item Mahasiswa dapat menunjukkan pengetahuan mereka tentang operator dalam bahasa C
	\item Mahasiswa mampu menggunakan fungsi untuk membaca masukan dari keyboard
	\item Mahasiswa mampu menggunakan fungsi untuk mencetak teks di layar

\end{itemize}
\section{Mengenal Bahasa C}

Bahasa C dikembangkan oleh Dennis M. Ritchie dan Brian W. Kernighan pada awal tahun 1970.\\
Terdapat beberapa standar untuk bahasa C, standar di sini dapat diartikan sebagai guideline dalam menuliskan bahasa C. Beberapa Standar yang ada:
\begin{enumerate}
	\item  Definisi Kernighan \& Ritchie (K\&R)
	\item ANSI-C (X-3.159 -1989-)
	\item Definisi AT\&T (untuk superset C, C++), dan
	\item GNU Coding Standards.
\end{enumerate}
\subsection*{}Aplikasi atau kegunaan Bahasa C
\begin{enumerate}
	\item     Membuat sistem operasi dan program-program sistem.
	\item Pemrograman yang "dekat" ke perangkat keras (misalnya untuk kontrol peralatan).
	\item Membuat tool kit.
	\item Menulis program aplikasi.
\end{enumerate}
\section{IDE (Integrated Development Environment)}
IDE adalah singkatan dari "Integrated Development Environment" dalam bahasa Inggris.
Dalam Bahasa Indonesia, IDE dapat diterjemahkan menjadi "Lingkungan Pengembangan Terintegrasi" atau "Ruang Kerja Pengembangan Terpadu."
IDE adalah sebuah perangkat lunak yang dirancang untuk membantu pengembang perangkat lunak dalam proses pengembangan, pengkodean, dan pengujian aplikasi komputer.
\\
Berikut ini adalah daftar aplikasi IDE bahasa C yang dapat digunakan.
\begin{itemize}
	\item CodeBlocks
	\item DevC++
\end{itemize}
\section{Membuat proyek baru pada IDE Code::Blocks}
\subsection{Langkah untuk membuat proyek baru}
\begin{enumerate}
	\item Go to File $>$ New $>$ Project
	      \begin{figure}[H]
		      \centering
		      \includegraphics[width=0.7\linewidth]{P1/img/screenshot002.png}
		      \caption{}
		      \label{fig:screenshot002}
	      \end{figure}
	\item Klik Console Application
	      \begin{figure}[H]
		      \centering
		      \includegraphics[width=0.7\linewidth]{P1/img/screenshot004.png}
		      \caption{}
		      \label{fig:screenshot004}
	      \end{figure}
	\item Pilih C sebagai bahasa Pemrograman
	      \begin{figure}[H]
		      \centering
		      \includegraphics[width=0.7\linewidth]{P1/img/screenshot005.png}
		      \caption{}
		      \label{fig:screenshot005}
	      \end{figure}
	\item Berikan nama ke project
	      \begin{figure}[H]
		      \centering
		      \includegraphics[width=0.7\linewidth]{P1/img/screenshot006.png}
		      \caption{}
		      \label{fig:screenshot006}
	      \end{figure}
	\item  Pilih compiler (gcc), pilih dirketori untuk menyimpan, dan klik save.
	      \begin{figure}[H]
		      \centering
		      \includegraphics[width=0.7\linewidth]{P1/img/screenshot007.png}
		      \caption{}
		      \label{fig:screenshot007}
	      \end{figure}
	\item Ketikan kode pada \ref{fig:screenshot008} ke Code::Blocks
	      \begin{figure}[H]
		      \centering
		      \includegraphics[width=0.7\linewidth]{P1/img/screenshot008.png}
		      \caption{}
		      \label{fig:screenshot008}
	      \end{figure}
	\item Klik Build$->$Build and Run atau tekan F9
	      \begin{figure}[H]
		      \centering
		      \includegraphics[width=0.7\linewidth]{P1/img/screenshot009.png}
		      \caption{}
		      \label{fig:screenshot009}
	      \end{figure}
	      % \item The program outputs can be seen on the console.
	\item Keluaran dari program dapat dilihat di console
	      \begin{figure}[H]
		      \centering
		      \includegraphics[width=0.7\linewidth]{P1/img/screenshot010.png}
		      \caption{}
		      \label{fig:screenshot010}
	      \end{figure}
\end{enumerate}

% \subsection{Exercise}
% Create a project with name HaloDunia and write the program like can be seen on figure \ref{fig:screenshot008} but change \verb|Hello World!| with \verb|Halo Dunia!|
\subsection{Tugas Pendahuluan}
\begin{enumerate}
	\item Buatlah proyek dengan nama HaloDunia dan tulis program seperti gambar \ref{fig:screenshot008} ubahlah \verb|Hello World!| dengan kata lain!
\end{enumerate}

%\begin{enumerate}
%	\item  Membuat program untuk menampilkan tulisan ke layar.\\
%	Langkah-langkah
%	\begin{enumerate}
%		\item Buatlah project baru dengan nama :\verb*|MencetakTextKeLayar|
%		\item Ketiklah ulang kode pada Listing \ref{lst:mencetaksalamkelayar}
%		\begin{figure}[H]
%		\begin{lstlisting}[language=c,label=lst:mencetaksalamkelayar,caption=Mencetak Teks Kelayar,captionpos=t]
%			/*Mencetak Text ke layar*/
%			
%			#include <stdio.h>
%			
%			int main()
%			{
%				//Mencetak ke layar
%				printf("Saya belajar  Pemprograman Komputer\n");
%				return 0;
%			}
%			
%		\end{lstlisting}
%	\end{figure}
%	\end{enumerate}
%\end{enumerate}
% \section{Structure of C Programming Language}
\section{Struktur Bahasa Pemrograman C}

\begin{lstlisting}[language=c,caption=Contoh program sederhana dalam bahasa C,label=lst:helloworld,captionpos=t]
#include <stdio.h>

int main()
{
	//printing to screen
	printf("Halo Dunia");
	return 0;
}
\end{lstlisting}

% Code on Listing \ref{lst:helloworld} is a simple program to print "Halo Dunia" to screen. The following is the explanation what each line of code do in the program.
Kode pada listing \ref{lst:helloworld} adalah program mudah untuk mencetak "Halo Dunia" ke layar. Berikut penjelasan yang dilakukan setiap baris kode pada program.
\begin{itemize}\setlength\itemsep{-0.1em}
	% \item [Baris 1 :] \verb|#include <stdio.h>|\\ header file library for input and output functions like \verb|printf()| (the one used on line 6)
	% \item[Baris 2 :] Empty line. 
	% \item [Baris 3 :] \verb|int main()|\\ The main function. The main function is the first function to be ran when the program starts.
	% \item[Baris 4 :] \{ \\Beginning of the \verb|main()| function code block.
	% \item[Baris 5 :]\verb|//printing to screen|\\ Comments. Comments are used to explain what the program is doing. Comments are ignored by the program, but helps the reader.
	% \item[Baris 6 :]\verb|printf("Halo Dunia");|\\ Printing "Halo Dunia" to the screen.
	% \item[Baris 7 :] \verb|return 0;| \\Returning the \verb|main()| function (A function ends when it returns)
	% \item [Baris 8 :] \}\\Closing the \verb|main()| function code block.

	\item [Baris 1 :] \verb|#include <stdio.h>|\\ file library header untuk fungsi input dan output seperti \verb|printf()| (contoh digunakan di baris 6)
	\item[Baris 2 :] Baris kosong.
	\item [Baris 3 :] \verb|int main()|\\ Fungsi main. Fungsi utama adalah fungsi pertama yang akan dijalnkan ketika program dimulai.
	\item[Baris 4 :] \{ \\Permulaan dari \verb|main()| fungsi code block.
		\item[Baris 5 :]\verb|//printing to screen|\\ Komen. Komen digunakan untuk menjelaskan program. Komen akan diabaikan oleh program, tetapi membantu pembaca.
		\item[Baris 6 :]\verb|printf("Halo Dunia");|\\ Print/mencetak "Halo Dunia" ke layar.
	\item[Baris 7 :] \verb|return 0;| \\Mengembalikan \verb|main()| fungsi (sebuah fungsi berakhir ketika dikembalikan/return)
	\item [Baris 8 :] \}\\Menutup \verb|main()| fungsi code block.

\end{itemize}
% \subsection{Exercise}
% Try to swap line 6 and line 7 in Listing \ref{lst:helloworld}. What happened?\\
% What if \verb|return 0;| replaced with \verb|return 1;|?
\subsection{Tugas Pendahuluan}
\begin{enumerate}
	\item Cobalah untuk menukar baris 6 dan baris 7 di listing \ref{lst:helloworld}. Apa yang terjadi? Jelaskan!
	\item Bagaimana jika \verb|return 0;| diganti dengan \verb|return 1;|?
\end{enumerate}


\section{Tipe Data dan Variabel}
\subsection{Tipe Data}
Pada bahasa C, terdapat beberapa tipe data untuk merepresentasikan data yang berupa bilangan bulat, bilangan real, karakter, string, dan lain-lain. Berikut adalah beberapa tipe data pada C.
% In C programming language, there are several data types to represent integer, real number, characters, string, and etc.
\begin{center}
	\captionof{table}{Beberapa tipe data di C \label{tab:tipedata}}
	% \begin{tabular}{|l|l|l|}
	% 	\hline
	% 	Data Types & Size         & Description                                         \\ \hline
	% 	int       & 2 or 4 bytes & saves integers                        \\ \hline
	% 	float     & 4 bytes      & saves real numbers to 8 digit behind decimal point. \\ \hline
	% 	double    & 8 bytes      & saves real numbers to 15 digits behind decimal point. \\ \hline
	% 	char      & 1 byte       & saves a character                     \\ \hline
	% \end{tabular}
	\begin{tabular}{|l|l|l|}
		\hline
		Tipe Data                      & Ukuran   & Rentang Nilai                   \\ \hline
		Int (or signed int)            & 2 bytes  & -32,768 to 32,767               \\ \hline
		unsigned int                   & 2 bytes  & 0 to 65,535                     \\ \hline
		Short int(or signed short int) & 2 bytes  & -32,768 to 32,767               \\ \hline
		Long(or singed short int)      & 4 bytes  & -2,147,483,648 to 2,147,483,647 \\ \hline
		unsigned long                  & 4 bytes  & 0 to 4,294,967,295              \\ \hline
		float                          & 4 bytes  & 1.2E-38 to 3.4E+38              \\ \hline
		double                         & 8 bytes  & 2.3E-308 to 1.7E+308            \\ \hline
		Long double                    & 10 bytes & 3.4E-4932 to 1.1E+4932          \\ \hline
		char(or signed char)           & 1 byte   & -128 to 127                     \\ \hline
		unsigned char                  & 1 byte   & 0 to 255                        \\ \hline
	\end{tabular}
\end{center}
Untuk menampilkan data pada layar, setiap tipe data memiliki format specifier yang dapat digunakan pada formatted string. Berikut adalah format specifier untuk beberapa tipe data.
% To show the data on screen, every data type has a format specifier that can be used on formatted string. The following is the format specifier for several data types.
\begin{center}
	\captionof{table}{Format Specifier \label{tab:formatspecifier}}
	\begin{tabular}{|l|l|}
		\hline
		Format Specifier & Tipe Data \\ \hline
		\%d or \%i       & int       \\ \hline
		\%f              & float     \\ \hline
		\%lf             & double    \\ \hline
		\%c              & char      \\ \hline
		\%s              & string    \\ \hline
	\end{tabular}
\end{center}
Masih ada lebih banyak tipe data dari pada yang dituliskan pada Tabel \ref{tab:tipedata}. Tipe-tipe data ini dan spesifikasinya bisa ditemukan dengan mudah di internet.
% There are still more data types that what was written on Table \ref{tab:tipedata}. These data types and its specification can be found easily on the internet.

\subsubsection{Modifier}
Secara garis besar modifier merupakan sebuah kata, frasa atau klausa yang mengubah makna atau menjelaskan kata benda(noun) atau kata kerja (verb) dalam sebuah kalimat.
Sedangkan modifier dalam bahasa pemrograman dalah sebuah kata kunci atau keyword yang digunakan untuk mengubah perilaku atau sifat suatu elemen dalam program, seperti variabel, fungsi, atau kelas.
Kita menggunakan modifier untuk mengubah range dari tipe data dasar untuk menyesuaikan dengan keperluan pemrograman. Ada empat modifier, yaitu:
\begin{enumerate}
	\item signed \\
	      \verb|int value = -10;| (Menggunakan tanda negatif pada variabel bertipe int yang secara default adalah signed integer.)
	\item unsigned \\
	      \verb|unsigned int count = 100;|  (Menggunakan variabel unsigned integer untuk menyimpan bilangan bulat positif tanpa tanda.)
	\item long \\
	      \verb|long population = 7500000000;| (Menggunakan tipe data long untuk menyimpan nilai yang lebih besar daripada tipe data int.)
	\item short \\
	      \verb|short temperature = 20;| (Menggunakan tipe data short untuk menghemat memori ketika kita tahu bahwa nilai yang akan disimpan akan relatif kecil.)
\end{enumerate}


\subsection{Variabel}
Variabel merupakan tempat menyimpan data. Mendeklarasikan suatu variabel dapat dilakukan dengan cara sebagai berikut.
\begin{lstlisting}[language=c,caption=Deklarasi Variabel C,label=lst:deklarasivariabel,captionpos=t]
DataType VariableName;
\end{lstlisting}
\subsubsection{Operator Aritmatika dan Penugasan (Assignment)}
Operator penugasan dapat dilakukan pada variabel yang tidak mempunyai \verb*|const| atau merupakan variabel nilai-l. Namun operator aritmatika dapat menerima kedua variabel dengan const atau tidak (nilai-l dan nilai-r).
Tabel di bawah menunjukkan beberapa operator aritmatika di C.
\begin{center}
	\captionof{table}{Operator Aritmatika di C \label{tab:operatoraritmatika}}
	\begin{tabular}{|c|l|c|}
		\hline
		\multicolumn{1}{|l|}{\textbf{Operator}} & \textbf{Nama} & \multicolumn{1}{l|}{\textbf{Contoh}} \\ \hline
		+                                       & Penambahan    & \verb|x + y |                        \\ \hline
		-                                       & Pengurangan   & \verb|x = y|                         \\ \hline
		*                                       & Perkalian     & \verb|x * y|                         \\ \hline
		/                                       & Pembagian     & \verb|x/y|                           \\ \hline
		\%                                      & Modulo        & \verb|x % y|                         \\ \hline
	\end{tabular}
\end{center}

Tabel di bawah menunjukkan beberapa operator penugasan.
\begin{center}
	\captionof{table}{Operator Penugasan \label{tab:operatorpenugasan}}
	\begin{tabular}{|c|c|c|}
		\hline
		\multicolumn{1}{|l|}{Operator} & \multicolumn{1}{l|}{Contoh}       & \multicolumn{1}{l|}{Arti yang sama} \\ \hline
		=                              & x = 5                             & x = 5                               \\ \hline
		+=                             & x += 3                            & x = x + 3                           \\ \hline
		-=                             & x -= 3                            & x = x - 3                           \\ \hline
		*=                             & x *= 3                            & x = x * 3                           \\ \hline
		/=                             & x /= 3                            & x = x / 3                           \\ \hline
		\%=                            & x \%= 3                           & x = x \% 3                          \\ \hline
		\&=                            & x \&= 3                           & x = x \& 3                          \\ \hline
		|=                             & x |= 3                            & x = x | 3                           \\ \hline
		\textasciicircum{}=            & x \textasciicircum{}= 3           & x = x \textasciicircum 3            \\ \hline
		\textgreater{}\textgreater{}=  & x \textgreater{}\textgreater{}= 3 & x = x \textgreater{}\textgreater 3  \\ \hline
		\textless{}\textless{}=        & x \textless{}\textless{}= 3       & x = x \textless{}\textless 3        \\ \hline
	\end{tabular}
\end{center}
Terdapat juga "singkatan" untuk beberapa operator penugasan seperti \verb*|x+=1| dan \verb*|x-1| yaitu \verb*|++| dan \verb*|--| yang masing-masing disebut kenaikan dan penurunan.
Singkatan ini digunakan seperti berikut.
\begin{verbatim}
    x++;
    x--;
    ++x;
    --x;
\end{verbatim}

\subsubsection{Operator Bitwise}
Bitwise adalah operator khusus untuk menangani operasi logika bilangan biner dalam bentuk bit. \\
Bilangan biner sendiri merupakan jenis bilangan yang hanya terdiri dari 2 jenis angka, yakni 0 dan 1.
Jika nilai asal yang dipakai bukan bilangan biner, akan dikonversi secara otomatis oleh compiler C menjadi bilangan biner. Misalnya 7 desimal = 0111 dalam bilangan biner.
\\
\begin{center}
	\captionof{table}{Operator Bitwise \label{tab:operatorbitwise}}
	\begin{tabular}{|c|c|c|c|c|c|}
		\hline
		\multicolumn{1}{|l|}{Operator} & \multicolumn{1}{|l|}{Nama} & \multicolumn{1}{|l|}{Contoh}    & \multicolumn{1}{|l|}{Biner}       & \multicolumn{1}{|l|}{Hasil (biner)} & \multicolumn{1}{|l|}{Hasil (decimal)} \\ \hline
		\&                             & AND                        & 10 \& 12                        & 1010 \& 1100                      & 1000                                & 8                                     \\ \hline
		|                              & OR                         & 10 | 12                         & 1010 | 1100                       & 1110                                & 14                                    \\ \hline
		\textasciicircum{}             & XOR                        & 10 \textasciicircum 12          & 1010 \textasciicircum 1100        & 0110                                & 6                                     \\ \hline
		$\sim$                         & NOT                        & $\sim$10                        & $\sim$1010                        & 0101                                & -11 (two complement)                  \\ \hline
		\textless{}\textless{}         & Left shift                 & 10 \textless{}\textless 1       & 1010 \textless{}\textless 1       & 10100                               & 20                                    \\ \hline
		\textgreater{}\textgreater{}   & Right shift                & 10 \textgreater{}\textgreater 1 & 1010 \textgreater{}\textgreater 1 & 101                                 & 5                                     \\ \hline
	\end{tabular}
\end{center}

\begin{center}
	\colorbox{pink}{\parbox{0.8\linewidth}{\textbf{Catatan:} Terdapat beberapa operator dalam Bahasa C. Silahkan pelajari dengan mencari referensi secara mandiri.}}
\end{center}
\subsection{Tugas Pendahuluan}
\begin{lstlisting}[language=c,caption=Menggunakan operator penugasan pada variabel const,label=lst:constassignment,captionpos=t]
#include <stdio.h>
int main()
{
    //deklarasi variabel const
    const int x=0;
    x=1;
	return 0;
}
\end{lstlisting}
% Try to compile the program in Listing\ref{lst:constassignment}, what happened?
Coba jalankan program di Listing \ref{lst:constassignment}, apa yang terjadi?


\section{Output dan Input}

\subsection{printf()}
\verb*|printf|  digunakan untuk mencetak string  ke output yang dilengkapi dengan format specifier yang dimulai dengan \verb*|%| pada string.
% is a function in C that is used to print formatted string.
% You can use format specifier within the formatted string to outputs your variables.

\begin{verbatim}
	printf(const char *format,v1,v2,..,vn)
\end{verbatim}

Format specifier untuk beberapa tipe data dapat dilihat pada Tabel \ref{tab:formatspecifier}
% The format specifier for each data types can be seen on Table \ref{tab:formatspecifier}


\begin{description}
	\item[Contoh \thesubsection.1]  Mencetak teks ke layar.
		\begin{lstlisting}[language=c,caption = Mencetak Tulisan "C Programming" Ke layar,captionpos=t]
		#include <stdio.h>    
		int main()
		{ 
			// Mencetak teks yang ditulis dalam simbol "
			printf("C Programming");
			return 0;
		}
	\end{lstlisting}
		\begin{itemize}
			\item Seluruh program C harus berisi fungsi main() tempat program memulai menjalankan kode.
			      % \item All C program must have main() function where the program needs to run the code.
			\item Fungsi \verb*|printf()| adalah library untuk mengirim output yang telah diformat ke layar.  Fungsi \verb*|printf()|  mencetak string dalam tanda dua tanda petik.
			      % \item \verb*|printf()| function is a function from stdio.h library. This function outputs the string inside the symbol " to the screen.
			      % \item \verb*|return 0;| statement in the \verb*|main()| function tells the program to exit.
			\item \verb*|return 0;| pernyataan di \verb*|main()| fungsi memberitahu program untuk keluar.
		\end{itemize}
	\item [Contoh \thesubsection.2] Mencetak integer.
	      \begin{lstlisting}[language=c,captionpos=t]
		#include <stdio.h>
		int main()
		{
			int testInteger = 5;
			printf("Number = %d", testInteger); // <- %d format string
			return 0;
		}
	\end{lstlisting}


	      Pada contoh ini digunakan format specifier \verb*|%d| untuk mencetak tipe data \verb*|int|. \verb*|%d| pada tex akan digantikan oleh isi dari \verb*|testInteger|.
	      % The code above uses the format specifier \verb*|%d| to prints \verb*|int| data type. The \verb*|%d| part of the string will be replaced with the value of \verb*|testInteger|.

	\item[Contoh \thesubsection.3] Keluaran bilangan real (float atau double)
		\begin{itemize}\label{eq:LuasSegitiga}
			\item \verb|Base|  : menggunakan \verb|float| tipe data.
			\item \verb|Height|: menggunakan \verb|float| tipe data.
			\item \verb|Area|  : menggunakan \verb|float| tipe data.
			      \begin{equation}
				      Luas = \frac{1}{2} \times Alas \times Tinnggi
			      \end{equation}
		\end{itemize}
		\begin{lstlisting}[language=c,captionpos=t]
		#include <stdio.h>
		
		int main()
		{
			// deklarasi variabel
			float Base;
			float Height;
			float Area;
			// inisialisasi nilai
			Base = 10;
			Height = 5;
			// menghitung area
			Area = 0.5*Base*Height;
			// mencetak area ke layar
			printf("Area = %f",Area);
			return 0;
		}
		
	\end{lstlisting}

		Penjelasan
		\begin{description}
			\item[Baris 6-8]  \verb|Alas|, \verb|Tinggi| dan \verb|Luas| bertipe data \verb|float| untuk menyimpan data parameter luas segitiga.
			\item[Baris 10 dan 11] Memberi nilai ke Variabel \verb|Alas|=10 dan \verb|Tinggi|=5
			\item[Baris 13] Menghitung luas alas sesuai dengan persamaan \ref{eq:LuasSegitiga}
			\item[Baris 15] Mencetak \verb|Luas| ke layar dengan menggunakan perintah \verb|printf|.
		\end{description}
\end{description}

.\subsection{scanf}
Fungsi  \verb*|scanf(const char *format, ...)| membaca input dengan format.
% \verb*|scanf(const char *format, ...)| reads input according to the format string.

\begin{enumerate}
	\item Syntax
	      \begin{verbatim}
		scanf(const char *format, ...)
	\end{verbatim}
	\item Parameter \\
	      % Format string in C consist of one or more whitespace, non-whitespace, and format specifiers.
	      Format string pada C yang terdiri dari satu atau lebih yang terdiri dari \\
	      Karakter Whitespace,Karakter Non-whitespace  dan  Format specifiers.
	\item Return Value \\
	      Ketika berhasil maka fungsi mengembalikan jumlah item dari argumen yang berhasil di baca.
	      % The function will return the number of arguments it has sucessfully read.

\end{enumerate}

\begin{description}
	\item  [Contoh \thesubsection.4] Menghitung luas segitiga dengan  dengan alas \verb*|Alas|   dan tinggi \verb*|Tinggi| yang diinputkan dari keyboard.
	      \begin{lstlisting}[language=c]
#include <stdio.h>

int main()
{
	float Alas ,Tinggi,Luas;
	
	printf("Menghitung luas segitiga\n");
	printf("\nMasukkan Alas= ");
	scanf("%f",&Alas);
	printf("\nMasukkan Tinggi=");
	scanf("%f",&Tinggi);
	Luas = 0.5*Alas *Tinggi;
	printf("Luas Seigtiga = %.2f", Luas);
	return 0;
}
	\end{lstlisting}
	      \begin{figure}[H]
		      \centering
		      \includegraphics[width=0.5\linewidth]{P1/img/screenshot0005.png}
		      \caption{}
		      \label{fig:screenshot0005}
	      \end{figure}

	      \begin{description}
		      \item [Baris 9]\verb|scanf("%f",&Alas);| meminta masukan untuk alas segitiganya
		      \item [Baris 11]\verb|scanf("%f",&Tinggi);| meminta masukan tinggi segitiganya
		      \item [Baris 13]\verb|printf("Triangle Area = %.2f", Luas);|,  \verb|.2| si \verb|%.2f| menandakan bahwa hanya 2 angka di belakang koma(decimal point) yang perlu dicetak.
	      \end{description}

	\item[Contoh \thesubsection.5] Program untuk memasukkan nama dan email.\\
		Pada contoh ini dipelajari bagaimana cara menginputkan string atau text dari keyboard dan mencetak kelayar. Input dari contoh program ini ada dua yang terdiri dari \verb|snama| dan \verb|sAlamatEmail|. Oleh karena text berisi banyak karakter maka masing-masing variabel dideklarasikan sebagai kumpulan karakter dengan jumlah karakter untuk sNama=20 dan sAlamatEmail=30.
		% This example shows how to input string or text from keyboard and outputs it on the screen. Input from this program consist of \verb|sName| and \verb|sEmail|.
		\begin{figure}[H]
			\begin{lstlisting}[language=c]
		#include <stdio.h>
		
		int main () 
		{
			char sName[20], sEmail[30];
			
			printf("Masukkan Nama: ");
			scanf("%19s", sName);
			
			printf("Masukkan Email: ");
			scanf("%29s", sEmail);
			
			printf("Nama : %s\n", sName);
			printf("Email:%s", sEmail);
			return(0);
		}
	\end{lstlisting}
		\end{figure}
\end{description}


\subsection{Escape Sequence}
Escape Sequence adalah urutan karakter yang digunakan untuk memformat output dan tidak ditampilkan ketika dicetak ke layar. Setiap karakter mempunyai fungsi tertentu.
% Some characters can't be written on the format string because they are used to format the outputs.
% So, to outputs those special characters we use escape sequences.

\begin{table}[H]
	\centering
	\captionof{table}{Escape Sequence \label{tab:escapesequence}}
	\begin{tabular}{|l|l|l|}
		\hline
		Escape sequence                  & Output berupa   \\ \hline
		\textbackslash{}a                & bell, alarm     \\ \hline
		\textbackslash{}b                & Backspace       \\ \hline
		\textbackslash{}f                & Ganti halaman   \\ \hline
		\textbackslash{}n                & Ganti baris     \\ \hline
		\textbackslash{}r                & Carriage return \\ \hline
		\textbackslash{}t                & tab horisontal  \\ \hline
		\textbackslash{}v                & tab vertikal    \\ \hline
		\textbackslash{}'                & Petik tunggal   \\ \hline
		\textbackslash{}"                & Petik Ganda     \\ \hline
		\textbackslash{}?                & Tanda tanya     \\ \hline
		\textbackslash{}\textbackslash{} & Backslash       \\ \hline
	\end{tabular}
\end{table}

\begin{description}
	\item[Contoh \thesubsection.6] Mengubah baris dengan escape sequence \verb*|\n|.
		\begin{lstlisting}
#include <stdio.h>

int main() 
{
	printf("Halo \nSaya sedang belajar bahasa C.\ndan ini sangat menyenangkan!");
	return 0;
}
	\end{lstlisting}
		\begin{figure}[H]
			\centering
			\includegraphics[width=0.5\linewidth]{P1/img/screenshot0006.png}
			\caption{}
			\label{fig:screenshot0006}
		\end{figure}

	\item[Contoh \thesubsection.7] Menggunakan escape sequence \verb*|\t| mengubah tab.
		\begin{lstlisting}[language=c]
#include <stdio.h>
int main(void)
{
	printf("Nama \t\t: Rahmad Rahardi\n");
	printf("Alamat \t\t: Bendungan Hilir Jakarta\n");
	printf("Tempat Lahir \t: Jakarta\n");
	printf("Tanggal Lahir \t: 30 Pebruari 2000\n");
	
	return (0);
}
\end{lstlisting}
		\begin{figure}[H]
			\centering
			\includegraphics[width=0.5\linewidth]{P1/img/screenshot0007.png}
			\caption{}
			\label{fig:screenshot0007}
		\end{figure}
\end{description}

\subsection{Tugas Pendahuluan}
\begin{enumerate}
	\item Cobalah buat suatu program yang dapat menerima input berupa nama dan NRP kemudian menampilkannya pada layar.
	\item Buat program yang meminta pengguna memasukkan suhu dalam Celsius dan kemudian mengonversinya ke Fahrenheit.
\end{enumerate}

\begin{center}
	\colorbox{cyan!30}{\parbox{0.8\linewidth}{\textbf{Opsional:} Pelajari Git dan Github. Anda dapat memulai pembelajaran dari sumber berikut ini: \\ \href{https://github.com}{GitHub - https://github.com} \\ \href{https://git-scm.com/doc}{Git -https://git-scm.com/doc}}}
\end{center}
% \section{Tujuan}
\begin{itemize}[label=$\bullet$, itemsep=-1pt, leftmargin=*]
	\item Mahasiswa mengenal dan mampu menggunakan ekspresi-ekspresi logika dan perbandingan pada bahasa pemrograman C
	\item Mahasiswa mengenal dan mampu menggunakan syntax-syntax percabangan pada bahasa pemrograman C
	\item Mahasiswa dapat mengenal dan menggunakan perulangan while pada bahasa C
	      % \item Students are able to use while-loop on C
	\item Mahasiswa dapat mengenal dan menggunakan perulangan do-while pada bahasa C
	      % \item Students are able to use do-while loop on C
	\item Mahasiswa dapat mengenal dan menggunakan perulangan for pada bahasa C
	      % \item Students are able to use for loop on C
	\item Mahasiswa dapat mengenal dan menggunakan  array dimensi satu maupun multidimensi.
	      % \item Students are able to use one dimensional or multidimensional array
	\item Mahasiswa mampu memanfaatkan perulangan untuk mengolah data pada array.
	      % \item Students are able to use loops to process data on arrays
	\item Mahasiswa  dapat mengenal dan menggunakan  string.
\end{itemize}
\section{Ekspresi Logika dan Perbandingan}
\subsection{Ekspresi Perbandingan}
Berikut adalah operator-operator yang digunakan pada suatu ekspresi perbandingan
% The following are the operators used in comparison expressions.
\begin{center}
	\captionof{table}{Operator Perbandingan \label{tab:operatorcomp}}
	\begin{tabular}{|c|l|c|}
		\hline
		Operator        & Nama                    & \multicolumn{1}{l|}{Contoh Ekspres} \\ \hline
		==              & Sama Dengan             & x == y                              \\ \hline
		!=              & Tidak Sama Dengan       & x != y                              \\ \hline
		\textgreater{}  & Lebih Dari              & x \textgreater y                    \\ \hline
		\textless{}     & Kurang Dari             & x \textless y                       \\ \hline
		\textgreater{}= & Lebih Dari Sama Dengan  & x \textgreater{}= y                 \\ \hline
		\textless{}=    & Kurang Dari Sama Dengan & x \textless{}= y                    \\ \hline
	\end{tabular}
\end{center}

Suatu ekspresi perbandingan akan mengembalikan nilai berupa \verb|true| atau \verb|false| yang ditandakan dengan nilai 0 atau 1.
% A Comparison Expression will return boolean value \verb|true| or \verb|false| which is also represented with the value 1 or 0.
Sebagai contoh:
As example:
\begin{verbatim}
    printf("%d",0>1); // Akan Mencetak 0 ke layar
    printf("%d",0<1); // Akan Mencetak 1 ke layar 
\end{verbatim}

\subsection{Ekspresi Logika}
Berikut adalah operator-operator logika yang digunakan pada suatu ekspresi logika
% The following are the logical operators used on a Logical Expression
\begin{center}
	\captionof{table}{Ekspresi Logika \label{tab:operatorlogic}}
	\begin{tabular}{|c|l|c|}
		\hline
		Operator & \multicolumn{1}{c|}{Nama} & Contoh Ekspresi        \\ \hline
		$\&\&$   & AND                       & $x<5\; \&\& \;x<10$    \\ \hline
		$||$     & OR                        & $x < 5\; ||\; x < 4  $ \\ \hline
		$!$      & NOT                       & $!(x <5 \&\& x < 10) $ \\ \hline
	\end{tabular}
\end{center}
Sama seperti ekspresi perbandingan, ekspresi logika akan mengembalikan nilai berupa true atau false
% Like comparison expression, logical expression will return boolean values.

\section{Percabangan}
\subsection{Pernyataan If}
% \verb*|if| statement is used to decide which block of code to be executed if the condition is true.
\verb*|if| digunakan untuk menentukan blok kode C yang dijalankan apabila ekspresi kondisi bernilai benar (TRUE),
\begin{verbatim}
// Block of code before if
if (Condition) 
{
 // Blok kode yang akan dieksekusi jika kondisinya benar(True).
}
// Blok kode setelah if
\end{verbatim}
Sebagai contoh, perhatikan program berikut
% As example, look at the following code 
\begin{lstlisting}[language=c,caption =Contoh Pernyataan If,label=lst:ifexample01]
	include <stdio.h>
	
	int main()
	{
		//Deklarasi variabel 
		int uangSaya,hargaRoti;
		uangSaya = 5000;
		hargaRoti = 10000;
		
		if (uangSaya>=hargaRoti)
		{
		    printf("saya bisa beli roti\n");
		}
		printf("hehe");
		return 0;
	}
\end{lstlisting}
Keluaran program ini
\begin{verbatim}
    hehe
\end{verbatim}
% If line 7 changed to \verb|uangSaya=10000|, the outputs of the program would be
Jika baris ke 7 diganti dengan \verb|uangSaya=10000| maka output dari program ini akan menjadi
\begin{verbatim}
    saya bisa beli roti
    hehe
\end{verbatim}

\subsection{Pernyataan If-else}
Pernyataan else digunakan untuk menentukan blok kode yang di jalankan apabila kondisi salah.
% Else statement is used to decide the block of code to be executed if the condition is false.
\begin{verbatim}
// Blok kode sebelum if
if (Condition) 
{
	// Blok kode yang akan dieksekusi jika kondisinya benar
} else
{
	// Blok kode yang akan dieksekusi jika kondisinya salah
}
// Blok kode setelah pernyataan if-else
\end{verbatim}
Berikut contoh penggunaan if-else
% The following is an example of using if-else statement:
\begin{lstlisting}[language=c,caption = if-else example,label=lst:ifelseexample01]
	include <stdio.h>
	
	int main()
	{
		//Deklarasi variabel 
		int uangSaya,hargaRoti;
		uangSaya = 5000;
		hargaRoti = 10000;
		
		if (uangSaya>=hargaRoti)
		{
		    printf("saya bisa beli roti\n");
		}
		else
		{
	        printf("saya tidak bisa beli roti\n");	
		}
		printf("hehe");
		return 0;
	}
\end{lstlisting}
Keluaran program adalah sebagai berikut
\begin{verbatim}
    saya tidak bisa beli roti
    hehe
\end{verbatim}
Jika baris ke 7 diganti dengan \verb|uangSaya=10000| maka output dari program ini akan menjadi
% If line 7 changed to \verb|uangSaya=10000|, the outputs of the program would be
\begin{verbatim}
    saya bisa beli roti
    hehe
\end{verbatim}

\subsection{Pernyataan if-else if}
Statement \verb|else if| digunakan untuk menjalankan blok kode apabila kondisi statement \verb|if| atau \verb|else if| sebelumnya bernilai salah.
% The \verb|else if| statement is used to run a block of code when the condition in \verb|if| or the previous \verb|else if| is false.
\begin{verbatim}
	// blok kode sebelum if
    if (Condition1)
    {
	  /* blok kode yang akan dieksekusi jika Kondisi 1
	  adalah benar*/
    }
    else if (Condition2)
    {
	  /* blok kode yang akan dieksekusi jika Kondisi 1 salah
	  dan Kondisi 2 benar */
    }
    else if (Condition3)
    {
	  /* Blok kode yang akan dieksekusi kapan
	  Kondisi 1 dan Kondisi 2 salah dan
	  Kondisi 3 benar*/
    }
    ...
    else if (ConditionN)
    {
	  /*Blok kode yang akan dieksekusi kapan
	  Kondisi 1 hingga KondisiN-1 salah dan
	  Kondisi N benar*/
    }
    else
    {
	  /* Blok kode yang akan dieksekusi kapan
	  Kondisi 1 hingga Kondisi N salah*/
    }
	// Blok kode setelah if
\end{verbatim}
Berikut contoh penggunaan if-else if
\begin{lstlisting}[language=c,caption = Contoh if-else if,label=lst:ifelseifexample01]
	include <stdio.h>
	
	int main()
	{
		//Deklarasi variabel 
		int uangSaya,hargaRoti;
		uangSaya = 5000;
		hargaRoti = 10000;
		
		if (uangSaya>hargaRoti)
		{
		    printf("saya bisa beli roti\n");
		}
		else if(uangSaya==hargaRoti)
		{
		    printf("saya bisa beli roti tapi uang saya akan langsung habis\n");
		}
		else
		{
	        printf("saya tidak bisa beli roti\n");	
		}
		printf("hehe");
		return 0;
	}
\end{lstlisting}
Output dari program ini adalah
\begin{verbatim}
    saya tidak bisa beli roti
    hehe
\end{verbatim}
Jika baris ke 7 diganti dengan \verb|uangSaya=10000| maka output dari program ini akan menjadi
% If line 7 changed to \verb|uangSaya=10000|, the output of the program would be
\begin{verbatim}
    saya bisa beli roti tapi uang saya akan langsung habis
    hehe
\end{verbatim}
Jika baris ke 7 diganti dengan \verb|uangSaya=12000| maka output dari program ini akan menjadi
% If line 7 changed to \verb|uangSaya=12000|, the output of the program would be
\begin{verbatim}
    saya bisa beli roti
    hehe
\end{verbatim}

\subsection{Nested if}
Nested if merupakan konsep di mana di dalam suatu blok if terdapat statement if.
% Nested if is when there is a conditional statements within a block of code inside the conditional statement
\begin{verbatim}
// Blok kode sebelum if
if (Condition1) 
{
    if (Condition2)
    {
        // lakukan sesuatu
    }
    else
    {
        // lakukan sesuatu yang lain
    }
} 
else
{
    // melakukan sesuatu yang lain
}
\end{verbatim}

Berikut contoh penggunaan nested if
% Below is an example of using nested if

\begin{lstlisting}[language=c,caption = Contoh nested if,label=lst:nestedifexample01]
	include <stdio.h>
	
	int main()
	{
		// Declare the variables
		int myMoney,breadPrice,friendsMoney;
		myMoney = 5000;
		breadPrice = 10000;
		friendsMoney = 42069;
		
		
		if (myMoney>breadPrice)
		{
		    printf("I can buy bread\n");
		}
		else if(myMoney==breadPrice)
		{
		    printf("I can buy bread but I will ran out of money\n");
		}
		else
		{
		    if(friendsMoney+myMoney >= breadPrice)
		    {
		        printf("I can buy bread if I borrow my friend money\n"); 
		    }
		    else
		    {
	            printf("I can't buy bread\n");	
		    }
		}
		printf("hehe");
		return 0;
	}
\end{lstlisting}


\subsection{Tugas Pendahuluan}
\begin{enumerate}
	\item Buatlah program yang menerima input 3 buah bilangan bulat A, B, dan C. Outputkanlah 3 bilangan bulat itu ke layar dengan urutan paling kecil ke paling besar. Lakukanlah ini dengan menggunakan statement if, if else, if else if, atau nested if.
	      %\item Try to make a program that receives 3 integer input A, B, and C. Then outputs those 3 integers to the screen sorted from smallest to largest. Do this only using conditional statements.	
\end{enumerate}

\section{Perulangan}
\subsection{Perulangan while}
Perulangan while akan menjalankan blok kode yang berada di dalamnya selama kondisi perulangan masih bernilai benar.
% While loop will run the code block within it repeatedly as long as the loop condition is true


\begin{figure}[H]
	\centering
	\includegraphics[width=0.4\linewidth]{P2/img/whileloop.png}
	\caption{Flow chart perulangan while}
	\label{fig:whileloop}
\end{figure}

Syntaxnya pada bahasa C adalah sebagai berikut:
% Its syntax in C programming language is as follows
\begin{verbatim}
    while(Condition)
    {
        // Blok kode yang akan diulang
    }
\end{verbatim}

Sebagai contoh, perhatikan kode berikut
% As an example, look at the following code
\begin{lstlisting}[language=c,caption = Contoh Penggunaan while,label=lst:whileexample01]
int main()
{
	int uangSaya,hargaRoti;
	uangSaya = 10000;
	hargaRoti = 2000;
	while(uangSaya >= hargaRoti)
	{
	    printf("Beli roti 1, uang saya sisa %d", uangSaya - hargaRoti);
	    uangSaya -= hargaRoti;
	}
	printf("Uang saya tidak cukup lagi");
	return 0;
}
\end{lstlisting}
Keluaran program di Listing \ref{lst:whileexample01} adalah sebagai berikut
\begin{verbatim}
    Beli roti 1, uang saya sisa 8000
    Beli roti 1, uang saya sisa 6000
    Beli roti 1, uang saya sisa 4000
    Beli roti 1, uang saya sisa 2000
    Beli roti 1, uang saya sisa 0
    Uang saya tidak cukup lagi
\end{verbatim}

Pada contoh ini, operasi pada baris 9 membuat variabel \verb|uangSaya| berkurang 2000 pada setiap pengulangan hingga akhirnya nilai \verb|uangSaya| tidak lebih dari atau sama dengan \verb|hargaRoti| lagi.
% You can see the line 9 of the code causes the variable \verb|uangSaya| to have its value substracted by 2000 for every loop until \verb|uangSaya| is no longer greater than equal to \verb|hargaRoti|. 
% The loop condition will be invalid and finaly exits the loop. Then it prints "Uang saya tidak cukup lagi", the command after the while loop statement.
Kondisi perulangan akan menjadi tidak valid dan akhirnya keluar dari perulangan. Kemudian ia mencetak "Uang saya tidak cukup lagi", perintah setelah pernyataan while loop.
\subsection{do-while loop}
do-while loop sebenarnya sama seperti while loop hanya saja do-while akan menjalankan perintah pada blok kode didalamnya terlebih dahulu sebelum melakukan pengecekan kondisi.
% do-while loop is very similar to while loop. The only difference is that do-while loop will execute the code block inside it once, and then checks the condition.
\begin{figure}[H]
	\centering
	\includegraphics[width=0.4\linewidth]{P2/img/dowhileloop.png}
	\caption{Pernyataan do-while}
	\label{fig:dowhileloop}
\end{figure}
Syntaxnya pada bahasa C adalah sebagai berikut:
% Its syntax in C is as follows:
\begin{verbatim}
    do{
        // the block of code that will be repeated
    }while(Condition)
\end{verbatim}
Sebagai contoh, perhatikan kode berikut
% Look at the following example.
\begin{lstlisting}[language=c,caption = Contoh Penggunaan do-while,label=lst:dowhileexample01]
int main()
{
	int uangSaya,hargaRoti;
	uangSaya = 10000;
	hargaRoti = 12000;
	do{
	    printf("Beli roti 1, uang saya sisa %d", uangSaya - hargaRoti);
	    uangSaya -= hargaRoti;
	}while(uangSaya >= hargaRoti)
	printf("Uang saya tidak cukup lagi");
	return 0;
}
\end{lstlisting}
Output dari program pada Listing \ref{lst:dowhileexample01} adalah
% The output of the code above are
\begin{verbatim}
    Beli roti 1, uang saya sisa -2000
    Uang saya tidak cukup lagi
\end{verbatim}
% The variable \verb|uangSaya| is substracted by \verb|hargaRoti| before checking the \verb|uangSaya>=hargaRoti| condition.
% Had the code above uses while loop, the repeating block of code wouldn't have executed even once.
Variabel \verb|uangSaya| dikurangi dengan \verb|hargaRoti| sebelum memeriksa \verb|uangSaya>=hargaRoti| kondisi.
Seandainya kode di atas menggunakan perulangan while, blok kode yang berulang tidak akan dieksekusi sekali pun.
\subsection{Perulangan for}
Misalkan terdapat blok kode while dengan bentuk seperti ini:
% If you have a block of code like this:
\begin{verbatim}
    InitializationStatement; // e.g.: int i = 0;
    while(Condition){
        // do something
        updateStatement; // e.g.: i++ 
    }
\end{verbatim}
Hal ini setara dengan
\begin{verbatim}
    for(InitializationStatement;Condition;updateStatement){
        // do something
    }
\end{verbatim}

Sebagai contoh, perhatikan program berikut:
% As example, look at the following code:
\begin{lstlisting}[language=c,caption = Contoh Penggunaan for,label=lst:forexample01]
int main()
{
    int i=0;
    for(i=1;i<10;i++){
        printf("%d ",i);
    }
	return 0;
}
\end{lstlisting}
Output dari program ini adalah
% The output of this program are
\begin{verbatim}
    1 2 3 4 5 6 7 8 9 
\end{verbatim}
Berikut kode pada Listing \ref{lst:forexample01} jika diubah menjadi bentuk while-loop
% The following is the code if code in Listing \ref{lst:forexample01} converted to its while-loop form
\begin{lstlisting}[language=c,caption = For dalam bentuk while,label=lst:forwhileform01]
int main()
{
    int i=0;
    i=1;
    while(i<10){
        printf("%d ",i);
        i++;
    }
	return 0;
}
\end{lstlisting}
\begin{center}
	\colorbox{pink}{\parbox{0.8\linewidth}{\textbf{Catatan:} Terdapat keyword break dan continue digunakan untuk mengendalikan (kontrol) alur pada perulangan. Pelajari secara mandiri!}}
\end{center}

\subsection{Tugas Pendahuluan}
\begin{enumerate}
	\item Implementasikan program dalam bahasa C yang menghitung faktorial dari sebuah bilangan bulat non-negatif yang dimasukkan oleh pengguna menggunakan loop do-while. Tampilkan hasilnya.
	\item Implementasikan program dalam bahasa C untuk mencari bilangan prima antara 1 dan 100. Gunakan loop for untuk mengiterasi melalui semua angka dan pernyataan continue untuk mengabaikan angka yang bukan prima. Tampilkan semua bilangan prima yang ditemukan.
\end{enumerate}

\section{Array}
Array atau biasa disebut larik adalah koleksi data dimana setiap elemen mempunyai nama yang sama dan bertipe sama. Setiap elemen diakses berdasarkan  indeks elemennya.
% Array is a collection of data where each element of it has the same name(indexed) and data type. Every element in an array can be accessed using its element index.
\subsection{Array 1D}
Variabel array dimensi satu dideklarasikan dengan menentukan jenis elemen dan jumlah elemen yang di perlukan oleh array.
% One dimensional array variable can be declared by deciding the data type of the element and the number of element that is needed.

Syntax:
\begin{verbatim}
    DataType variableName [arraySize];
\end{verbatim}
\begin{enumerate}
	\item \verb*|DataType|.\\
	      % The data type of the elements in the array, e.g. \verb|float|, \verb|int|, etc.
	      Jenis elemen data elemen array :\verb*|float|,\verb*|int|,\verb*|char| dsb
	\item \verb*|variableName|\\
	      Namariabel mengikuti aturan pemberian nama variabel,
	      % variableName follows the variable naming convention

	\item \verb*|arraySize| \\
	      % Integer more than 0. Defining the number of element an array has.
	      konstanta integer lebih besar dari 0. \\
\end{enumerate}

Untuk menginisialisasi array dimensi satu, dapat dilakukan dengan cara seperti berikut:
% Initializing one dimensional array can be done like shown below:
\begin{verbatim}
    int contoh_array[5] = {4,2,0,6,9};
\end{verbatim}

Data di dalam array dapat akses dengan menggunakan suatu bilangan yang merupakan index dari array tersebut. Perhatikan potongan kode berikut.
% Data in an array can be accessed by using an integer that is the index of the array. Look at the code below

\begin{lstlisting}[language=c,caption = Contoh Mengakses Array 1D,label=lst:array1d01]
int main()
{
    int arr[5] = {4,2,0,6,9};
    printf("%d\n",arr[0]);
    printf("%d\n",arr[4]);
    int i = 0;
    printf("%d\n",arr[i]);
    for(i=0;i<5;i++)
        printf("%d",arr[i]);
}
\end{lstlisting}

Potongan kode pada Listing \ref{lst:array1d01} akan memberikan output
% The code in Listing \ref{lst:array1d01} will give output
\begin{verbatim}
    4
    9
    4
    42069
\end{verbatim}

\subsection{Array 2D dan Array Multidimensi lainnya}%Array 2D dan Array Multidimensi lainnya}
Array dimensi dua pada dasarnya hanya merupakan array dimensi satu dari array dimensi satu. Oleh karena itu, untuk mendeklarasikan array dimensi dua kita dapat menggunakan syntax seperti berikut.
% 2D array is basically a 1D array of 1D array. Intuitively, you can define a 2D array like as seen below:
\begin{verbatim}
	DataType variableName[arraySize1][arraySize2];
\end{verbatim}
Hal ini berlaku juga untuk array dengan dimensi lebih dari dua.
% This also applies to multidimensional array.
\begin{verbatim}
    DataType variableName[arraySize1]...[arraySizeN];
\end{verbatim}
Akan ada $arraySize_1\times arraySize_2 \times \cdots \times arraySize_n$ elemen yang akan dialokasikan ke memori setelah melakukan array multidimensi seperti itu
% There will be $arraySize_1\times arraySize_2 \times \cdots \times arraySize_n$ of elements that would be allocated to the memory after doing multidimensional array like that.

Untuk menginisialisasi suatu array multidimensi dapat dilakukan sama seperti array biasa:
% To initialize multidimensional array, you can do the following:
\begin{verbatim}
    int arr[2][2] = {{1,2},{3,4}};
\end{verbatim}

\subsection{Tugas Pendahuluan}
\begin{enumerate}
	\item Cobalah inisialisasi suatu array multidimensi dengan menggunakan perulangan for.
	      % \item Try to initialize a multidimensional array with for loop
	\item Buatlah suatu program untuk mengisi data pada suatu array perdasarkan input dari keyboard.
	      % \item create a program to fill the data of an array by keyboard input.
	\item Apakah yang akan terjadi jika suatu array \verb|arr| diakses dengan \verb|arr[-1]|?
	      % \item What would happen if an array \verb|arr| is accessed with \verb|arr[-1]|?
	\item Apakah yang akan terjadi jika suatu array \verb|arr| dengan ukuran 5 diakses dengan \verb|arr[5]|?
	      % \item What would happen if an array \verb|arr| with size 5 is accessed with \verb|arr[5]|?
	\item Lihatlah kode berikut
	      % \item Look at the following code
	      \begin{verbatim}
        for(i=0;i<10;i++){
            for(j=i;j<10;j++){
                printf("A");
            }
        }
    \end{verbatim}
	      % How many "A" will be printed on the screen if that block of code is executed?
	      Ada berapa banyakah huruf A yang akan muncul pada layar jika program tersebut dijalankan?
\end{enumerate}

\section{String}
Secara umum, string merupakan kumpulan dari satu atau lebih karakter. Spesifik pada bahasa C, string didefinisikan sebagai kumpulan karakter yang diakhiri oleh karakter null \verb|'\0'|.
\\
Misalkan string  \verb|"Dasar"|, pada bahasa C direpresentasikan sebagai kumpulan karakter \verb|'D'|, \verb|'a'|, \verb|'s'|, \verb|'a'|, \verb|'r'|, dan \verb|'\0'|.

\subsection{Penggunaan String}
Karena string tidak lain adalah array dari char, maka cara pembuatan tipe data string dalam bahasa C juga sama seperti cara pembuatan array. Berikut contohnya:
\begin{lstlisting}[language=c,caption = Contoh String dari Char,label=lst:array1d01]
	#include <stdio.h>
 
	int main(void)
	{
	char foo[8] = {'b','e','l','a','j','a','r','\0'};
	printf("Isi variabel foo adalah %s \n", foo);
	
	return 0;
	}
\end{lstlisting}

\verb|‘\0’| adalah salah satu syarat pembuatan string di dalam bahasa C.
Semua string harus memiliki karakter “khusus” untuk menandakan akhir dari string.
Tanda \verb|‘\0’| mewakili karakter null yang dipakai oleh compiler bahasa C sebagai tanda akhir sebuah string.

Contoh source code penggunaan \verb|scanf| untuk membaca string:
\begin{lstlisting}[language=c,caption = Contoh String dengan scanf,label=lst:scanf]
	#include <stdio.h>

	int main() {
		// Mendeklarasikan variabel untuk menyimpan input dari pengguna
		int age;
		float height;
		char name[50];

		// Meminta pengguna untuk memasukkan usia mereka
		printf("Masukkan usia Anda: ");
		scanf("%d", &age);
		
		// Meminta pengguna untuk memasukkan tinggi mereka
		printf("Masukkan tinggi Anda (dalam meter): ");
		scanf("%f", &height);
		
		// Meminta pengguna untuk memasukkan nama mereka
		printf("Masukkan nama Anda: ");
		scanf("%s", name);

		// Menampilkan informasi yang dimasukkan pengguna
		printf("Nama: %s\n", name);
		printf("Usia: %d tahun\n", age);
		printf("Tinggi: %.2f meter\n", height);

		return 0;
	}
\end{lstlisting}


Contoh source code penggunaan \verb|gets| untuk membaca string:
\begin{lstlisting}[language=c,caption = Contoh String dengan gets,label=lst:gets]
#include <stdio.h>

int main () {
  
	char arr[100];
	while(true)
	{
		gets(arr);
		
		printf("-- %s\n", arr);
	}
  return 0;

}
\end{lstlisting}

String yang dibaca dengan mengunakan scanf atau gets akan secara otomatis memiliki \verb|null| character di akhir.

\subsection{Fungsi-Fungsi String}
Dalam bahasa pemrograman C, terdapat library yang dibuat dengan tujuan memudahkan pengguna dalam mengolah string.
Library tersebut tersimpan dalam \verb|<string.h>|,
oleh karena itu, untuk mengakses library ini, diperlukan tambahan preprocessor, yaitu:
\begin{lstlisting}[language=c]
	#include <string.h>
\end{lstlisting}

Pelajari berbagai fungsinya di \href{http://www.cplusplus.com/}{www.cplusplus.com}.

\subsection{Tugas Pendahuluan}
\begin{enumerate}
	\item Buatlah program dalam bahasa C yang mengambil dua string dari pengguna dan menentukan apakah kedua string tersebut anagram (mengandung karakter yang sama dalam urutan yang berbeda).
	      Tampilkan pesan yang sesuai.
	\item Jelaskan perbedaan antara string yang dideklarasikan sebagai array karakter (char array) dan string yang dideklarasikan sebagai tipe data string (string literal) dalam bahasa C. Berikan contoh penggunaan keduanya.
\end{enumerate}
% % \chapter{Fungsi (Subprogram)}
\section{Tujuan}
\begin{itemize}[label=$\bullet$, itemsep=-1pt, leftmargin=*]
    % \item Students understand how to create and call functions in C .
    \item Mahasiswa mengerti cara membuat dan memanggil fungsi pada bahasa pemrograman C.
          % \item Students able to pass parameter by value and by reference in C.
    \item Mahasiswa mampu menggunakan passing parameter by value dan by reference pada bahasa pemrograman C.
          % \item Students understand and able to apply recursion in C. 
    \item Mahasiswa mampu mengerti dan mengaplikasikan konsep rekursi pada bahasa pemrograman C.

\end{itemize}

\section{Fungsi}
Fungsi adalah sebuah kumpulan statement untuk melakukan tugas spesifik, yang bisa membutuhkan input ataupun tidak, untuk menghasilkan output yang sesuai.
% The advantages of using functions in C programming language are:
Keuntungan menggunakan fungsi pada bahasa pemrograman C adalah:
\begin{itemize}
    % \item Some code snippets can be reusable when using functions. 
    \item Beberapa cuplikan kode dapat digunakan kembali saat menggunakan fungsi.
          % \item We can call C functions any number of times in a program and from any place in a program.
          Kita dapat memanggil fungsi C berapa kali pun dalam suatu program dan dari mana saja dalam suatu program.
          % \item Large C codes can be splitted to several function, thus easier to track.
    \item Program c yang besar dapat dibagi ke dalam beberapa fungsi sehingga dapat dengan mudah untuk dilacak.
\end{itemize}
\subsection{Function Declaration}
% Every C program has atleast one function, that is the main() function. You can also define functions other than main().
Setiap program C mempunyai minimal satu fungsi, yaitu fungsi main(). Anda juga dapat mendefinisikan fungsi selain main()
Syntax :
\begin{verbatim}
    return_type function_name( parameters list){
        // function body
    	return something;
    }
\end{verbatim}
\begin{itemize}
    % \item Return Type.\\ The data type a function has to return.
    \item Return Type.\\ Tipe data yang harus dikembalikan suatu fungsi.
          % \item \verb*|function_name|.\\ The name of the function
    \item \verb*|function_name|.\\ Nama fungsi
          % \item parameters list.\\ 
          % The parameters of the function. 
    \item parameters list.\\
          Parameter dari fungsi.
          % \item Function body.\\ The block of code that will be executed when the function is called.
    \item Function body.\\ Kumpulan statemen yang mendefinisikan apa yang dilakukan oleh fungsi.
          % \item \verb|return something;|\\ A statement to return a value (\verb|something|). Returning the function causes the function to end.
          % For functions that doesn't return a value (\verb|void| type function), ending the function can be done by using \verb|return;|
    \item \verb|return something;|\\ merupakan statement untuk mengembalikan nilai dari fungsi. Untuk fungsi yang tidak mengembalikan nilai, dapat digunakan \verb|return_type| \verb|void|. Untuk keluar dari fungsi itu hanya perlu menggunakan statement \verb*|return|
\end{itemize}

% Example
Contoh

\begin{lstlisting}[language=c]
float TriangleArea(float Base, float Height)
{
	float Area;
	Area = 0.5*Base*Height;
	return Area;
}
\end{lstlisting}


% \subsection{Calling a Function}
\subsection{Memanggil Fungsi}

\begin{figure}[H]
    \centering
    \includegraphics[width=0.45\linewidth]{P3/img/screenshot005.png}
    \caption{}
    \label{fig:memanggilfungsi}
\end{figure}

\begin{lstlisting}[language=c]
	#include <stdio.h>
// Mendeklarasikan fungsi luasSegitiga
// Parameter input Alas , dan Tinggi
// Output float
float TriangleArea(float Base, float Height)
{
	float Area;
	Area = 0.5*Base*Height;
	return Area;
}
int main()
{
	float Bs = 4,Hg=10,L;
    // Memanggil Fungsi TriangleArea
	L=TriangleArea(Bs,Hg);
	printf("Area = %f",L);
	return 0;
}
\end{lstlisting}
\begin{enumerate}
    % \item Line 5-10: Defining the function \verb|Triangle Area| with 
    \item Baris 5-10:Mendefinisikan fungsi \verb*|TriangleArea| dengan
          \begin{itemize}
              % \item Two input parameter :\\
              \item 2 paramater input/masukan:
                    % input \verb*|Base| and \verb*|Height|  with \verb*|float| data type.
                    input \verb*|Base| dan \verb*|Height|  dengan tipe data \verb*|float|.
              \item Output bernilai tunggal dengan tipe data \verb*|float|.
          \end{itemize}
\end{enumerate}
% \subsection{Function with Arguments}
\subsection{Fungsi dengan Argumen}

\subsubsection{Argumen}


Jika suatu fungsi diharapkan untuk menggunakan argumen, maka variabel sebagai parameter yang menerima nilai dari argumen tersebut harus di dedeklarasikan terlebih dahulu. \\
\begin{enumerate}
    \item  \textbf{Parameter :}
          \begin{enumerate}
              \item Parameter adalah variabel dalam fungsi untuk merujuk ke salah satu bagian dari
                    data yang diberikan sebagai input ke fungsi.
                    % \item Parameters are the variable in the function that points to the part of the data that is inputted to the function.
              \item Data ini disebut argumen.
                    % \item These data is called arguments.

          \end{enumerate}

    \item \textbf{Formal Parameter:}
          \begin{enumerate}
              \item Parameter yang Ditulis dalam Definisi Fungsi Disebut "Parameter Formal".
                    % \item Parameter that is written within the function definition is called formal parameter.
              \item Parameter formal selalu variabel, sedangkan parameter aktual tidak harus variabel.
                    % \item Formal Parameter is always a variable, Actual Parameter however doesn't necessarily has to be a variable.

          \end{enumerate}


    \item \textbf{Actual Parameter:}
          \begin{enumerate}

              % \item Parameter that is used when calling the function
              \item Parameter yang Ditulis ketika memanggil fungsi.
                    % \item Actual Parameter could take the form of number, expression, or another function call.
              \item Dapat berupa angka, ekspresi, atau bahkan panggilan fungsi.
          \end{enumerate}
          \begin{figure}[H]
              \centering
              \includegraphics[width=0.5\linewidth]{P3/img/screenshot006.png}
              \caption{}
              \label{fig:parameterformalaktual}
          \end{figure}
\end{enumerate}
\subsection{Parameter Passing}
Passing parameter merupakan aktivitas menyalurkan nilai pada parameter saat memanggil fungsi. Pada umumnya, dikenal dua macam passing parameter yaitu:
% To use a function with parameter, the parameters must be passed to the function first.
% In general, there are two ways to pass paramater to a function
\begin{itemize}
    % \item Pass parameter by value, pass the value of the variable to the function.
    \item Pass parameter by value, yaitu menyalurkan \textbf{nilai} dari tiap parameter yang diberikan.
          % \item Pass parameter by reference, pass the reference of a variable (its memory address) to the function. 
    \item Pass parameter by reference, yaitu menyalurkan \textbf{alamat} dari tiap parameter yang diberikan.
\end{itemize}

\subsubsection{Passing Parameter by Value}

\begin{lstlisting}[language=c,caption = Passing by Value,label=lst:passbyvalue01]
    #include <stdio.h>

    int swapDanKembalikanJumlah(int x, int y) {
        int z;
        z = x;
        x = y;
        y = z;
        return x + y;
    }
    
    int main() {
        int a = 1;
        int b = 2;
        int jumlah = swapDanKembalikanJumlah(a, b);
        printf("Jumlah: %d\n", jumlah);
        printf("Nilai a dan b sekarang:\n");
        printf("a: %d\n", a);
        printf("b: %d\n", b);
        return 0;
    }
\end{lstlisting}

Perhatikan potongan kode pada Listing \ref{lst:passbyvalue01}. Baris 3-6 dari kode tersebut adalah operasi untuk menukar nilai dari 2 variabel. Namun, apabila program tersebut dijalankan, maka akan muncul output
% line 3-6 of the code in Listing \ref{lst:passbyvalue01} is a set of assignments to swap the values of 2 variable. However, when the program is executed, the output would be the following.
\begin{verbatim}
    Jumlah: 3
    Nilai a dan b sekarang:
    a: 1
    b: 2
\end{verbatim}
% The values of a and b did not swap. When passing parameter by value, anything that is done within the function body will have no effect on the parameter that is "passed on" the function. The value of the actual parameter will be assigned to the formal parameter, so we are not doing operation directly on the actual parameter.
Nilai dari a dan b tidak bertukar. Untuk passing parameter by value, apapun yang dilakukan pada function body tidak akan berpengaruh pada parameter yang "dipassingkan". Nilai dari parameter aktual akan diassign pada parameter formal.

\subsubsection{Passing Parameter by Reference}
Perhatikan baris 2 pada potongan kode berikut:
% Look at line 2 of the following code.
\begin{lstlisting}[language=c,caption = Passing by Reference,label=lst:passbyreference01]
#include <stdio.h>

void swap(int *x, int *y) {
    int z;
    z = *x;
    *x = *y;
    *y = z;
}

int main() {
    int a = 1;
    int b = 2;
    
    printf("Before swapping:\n");
    printf("a: %d\n", a);
    printf("b: %d\n", b);
    
    swap(&a, &b);
    
    printf("After swapping:\n");
    printf("a: %d\n", a);
    printf("b: %d\n", b);
    
    return 0;
}
\end{lstlisting}
Apabila program tersebut dijalankan, maka akan muncul output
% When this program is executed, it will output the following.
\begin{verbatim}
    Before swapping:
    a: 1
    b: 2
    After swapping:
    a: 2
    b: 1
\end{verbatim}

Ketika fungsi \verb|swapAndReturnTheSum(a,b)| dipanggil, alamat memori variabel a dan b "dipassingkan" pada fungsinya. Sehingga pada pada potongan kode di baris 4-6, x dan y akan mengacu pada memori parameter aktual yang dimasukkan di baris ke 13. Ketika melakukan passing by reference, kita tidak bisa memanggil fungsi dengan parameter yang tidak memiliki alamat memori. Sebagai contoh \verb|tukarDanKembalikansumnya(1,2)| tidak bisa dilakukan karena angka 1 dan 2 bukan variabel dan tidak memiliki alamat memori.
% When the function \verb|swapAndReturnTheSum(a,b)| is called, the memory address of \verb|a| and \verb|b| is passed on to the function. Therefore, in line 4-6, the \verb|x| and \verb|y| will point to the memory of the actual parameter that is inputted in line 13, so we are doing assignments directly to the actual parameter. When passing by reference, we can't call the function with parameter that has no memory address. As example \verb|swapAndReturnTheSum(1,2)| cannot be done as the number 1 and 2 doesn't have memory address.

\subsection{Tugas Pendahuluan}
\begin{enumerate}
    \item Buatlah fungsi yang dapat menerima 2 buah bilangan bulat a dan b kemudian mengembalikan nilai dari $a^b$
          % \item Create a function that can take 2 integer a and b then returns $a^b$
    \item Masalah-masalah apa yang akan lebih mudah diselesaikan dengan menggunakan fungsi?
          % \item What problems that can be solved easier with functions?
\end{enumerate}

\section{Rekursi}
Rekursi merujuk kepada definisi suatu hal yang dilakukan secara berulang-ulang.

Rekursi adalah ketika suatu fungsi dalam function bodynya memanggil fungsi itu sendiri.
% Recursion is when a function calls itself within its function body.
Sebagai contoh, perhatikan potongan kode berikut:
% As an example, look at the code below.
\begin{lstlisting}[language=c,caption = Factorial dengan rekursi,label=lst:recursionexample01]
int factorial(int n) {
    if (n==1)
        return 1;
    return n*factorial(n-1);
}
\end{lstlisting}
% The factorial function calls another factorial function in line 4.
Dapat dilihat bahwa fungsi factorial pada function bodynya memanggil factorial pada baris 4.
% Initialy, the function $factorial(n)$ is called. This function however will return 
Pada awalnya jika fungsi $factorial(n)$ dipanggil maka dia akan mencoba untuk mengembalikan
% $n\times factorial(n-1)$, then $factorial(n-1)$ akan mengembalikan $(n-1)\times factorial(n-1-1)$.
$n\times factorial(n-1)$, kemudian $factorial(n-1)$ akan mengembalikan $(n-1)\times factorial(n-1-1)$.
% Eventually it became like this:
Akhirnya menjadi seperti ini:
\begin{equation*}
    \begin{split}
        factorial(n)& = n \times factorial(n-1)\\
        & = n \times (n-1) \times factorial(n-2)\\
        & = n \times (n-1) \times (n-2) \times \cdots \times 2 \times factorial(1)\\
        & = n \times (n-1) \times (n-2) \times \cdots \times 2 \times 1\\
    \end{split}
\end{equation*}

\subsection{Tugas Pendahuluan}
\begin{enumerate}
    \item Diberikan sebuah baris bilangan 1, 5, 14, 30, ... dst. Buatlah sebuah program yang mengimplementasikan fungsi rekursif untuk menentukan bilangan ke-n dari pola tersebut.
\end{enumerate}
% \section{Tujuan}
\begin{itemize}[label=$\bullet$, itemsep=-1pt, leftmargin=*]
    \item Mahasiswa mengerti tentang konsep pointer pada bahasa pemrograman C.
    \item Mahasiswa mengerti cara membuat dan memanggil struct pada bahasa pemrograman C.
    \item Mahasiswa mengerti tentang algoritma sorting pada bahasa pemrograman C.
    \item Mahasiswa mengerti tentang algoritma searching pada bahasa pemrograman C.
    \item Mahasiswa mampu mengaplikasikan konsep algoritma searching dan sorting pada bahasa pemrograman C.
\end{itemize}

\section{Pointer}
\subsection{Alamat Memori}
Setiap variabel, fungsi, struct, ataupun objek lain yang dibuat dalam program mempunyai lokasi masing-masing pada memori. Alokasi setiap variabel disimpan dalam alamat memori tertentu.

Jika  terdapat variabel \verb|var| di program Anda, \verb|&var| akan memberi alamatnya di memori.
\begin{lstlisting}[language=c]
    int var = 5;
    printf("%d\n", var);
    printf("%p\n", &var);
\end{lstlisting}
\begin{center}
    \colorbox{pink}{\parbox{0.8\linewidth}{\textbf{Catatan:} Output bisa berbeda-beda di tiap eksekusi.}}
\end{center}

\subsection{Pengenalan Pointer}

Pointer (variabel penunjuk) adalah variabel khusus yang digunakan untuk menyimpan alamat, bukan nilai.

Deklarasi variabel pointer menggunakan operator \verb|*| di antara tipe data dan nama variabelnya.
\begin{lstlisting}[language=c]
	#include <stdio.h>
int main()
{
	int* p; // atau
    int * p2;
	return 0;
}
\end{lstlisting}


\subsection{Cara Kerja Pointer}
Berikut adalah cara kerja dari pointer.
\begin{lstlisting}[language=c, caption={Contoh Program Pointer}]
    #include <stdio.h>
    int main()
    {
       int* pc, c;
       
       c = 22;
       printf("Address of c: %p\n", &c);
       printf("Value of c: %d\n\n", c);  // 22
       
       pc = &c;
       printf("Address of pointer pc: %p\n", pc);
       printf("Content of pointer pc: %d\n\n", *pc); // 22
       
       c = 11;
       printf("Address of pointer pc: %p\n", pc);
       printf("Content of pointer pc: %d\n\n", *pc); // 11
       
       *pc = 2;
       printf("Address of c: %p\n", &c);
       printf("Value of c: %d\n\n", c); // 2
       return 0;
    }
\end{lstlisting}

Penjelasan Program:\\
\begin{enumerate}
    \item \verb|int* pc, c;|
          \begin{figure}[H]
              \centering
              \includegraphics[width=0.2\linewidth]{P4/img/screenshot001.png}
              \caption{}
              \label{fig:satu}
          \end{figure}
    \item \verb|c = 22;|
          \begin{figure}[H]
              \centering
              \includegraphics[width=0.2\linewidth]{P4/img/screenshot002.png}
              \caption{}
              \label{fig:dua}
          \end{figure}
    \item \verb|pc = &c;|
          \begin{figure}[H]
              \centering
              \includegraphics[width=0.2\linewidth]{P4/img/screenshot003.png}
              \caption{}
              \label{fig:tiga}
          \end{figure}
    \item \verb|c = 11;|
          \begin{figure}[H]
              \centering
              \includegraphics[width=0.2\linewidth]{P4/img/screenshot004.png}
              \caption{}
              \label{fig:empat}
          \end{figure}
    \item \verb|*pc = 2;|
          \begin{figure}[H]
              \centering
              \includegraphics[width=0.2\linewidth]{P4/img/screenshot005.png}
              \caption{}
              \label{fig:lima}
          \end{figure}
\end{enumerate}

\subsection{Double Pointer}
Variabel pointer juga dapat menunjuk variabel pointer lainnya.
Hal ini disebut dengan double pointer (pointer to pointer).
Untuk mendeklarasikan variabel double pointer, digunakan dua simbol *.
Kegunaan paling umum dari variabel double pointer adalah untuk membuat array dua dimensi secara dinamis.
\begin{lstlisting}[language=c]
    int **dbPtr;
\end{lstlisting}
Variabel dbPtr di atas menyimpan alamat memori dari variabel pointer lainnya. \\
Berikut contohnya
\begin{lstlisting}[language=c,  caption={Contoh Double Pointer}]
#include <stdio.h>

int main(void)
{
    int var = 23;
    int *ptr = &var;
    int **dbPtr = &ptr;

    printf("%d\n", **dbPtr);
        
    return 0;
}
\end{lstlisting}

\subsection{Tugas Pendahuluan}
\begin{enumerate}
    \item Bagaimana cara mendeklarasikan pointer ke array multidimensi?
    \item Buatlah program dalam bahasa C atau C++ yang mengimplementasikan
          fungsi void printMatrix(int **matrix, int rows, int cols) untuk mencetak matriks 2D menggunakan pointer ke pointer. Lalu, dalam fungsi main,
          buatlah matriks 2D dan panggil fungsi printMatrix untuk mencetak matriks tersebut.
\end{enumerate}

\section{Struct}
Dalam pemrograman C, struct (atau struktur) adalah kumpulan variabel (bisa dari tipe berbeda) di bawah satu nama.
Tidak seperti array yang hanya dapat menyimpan elemen dengan tipe data sama,
struct dapat mengelompokkan elemen dengan tipe data yang berbeda-beda.


\subsection{Deklarasi Struct}
Seperti variabel, struct harus dideklarasikan terlebih dahulu sebelum bisa digunakan. Pendeklarasian struct menggunakan sintaks sebagai berikut.

\begin{lstlisting}[language=c]
struct <nama_struct> {
    <tipe_data_member> <nama_member>;
    <tipe_data_member> <nama_member>;
    <tipe_data_member> <nama_member>;
    .
    .
    .
};
\end{lstlisting}

Berikut adalah contoh deklarasi struct berdasarkan kasus Mahasiswa.
\begin{lstlisting}[language=c]
struct Mahasiswa
{
    char *name;
    char *address;
    int age;
};
\end{lstlisting}
\begin{center}
    \colorbox{pink}{\parbox{0.8\linewidth}{\textbf{Catatan:}  Menggunakan pointer * untuk data string}}
\end{center}

Setelah dideklarasikan, sebuah struct akan menjadi tipe data baru.
Maka dalam kasus ini, struct Mahasiswa di sini menjadi tipe data baru dengan member-member berupa \verb|nama|, \verb|address|, dan \verb|age|.
Untuk membuat variabel dengan tipe data struct, dilakukan dengan sintaks berikut.

\begin{lstlisting}[language=c]
    struct <nama_struct> <nama_variabel>;
\end{lstlisting}

Contoh:
\begin{lstlisting}[language=c]
    struct Mahasiswa mhs1;
    struct Mahasiswa mhs2;
\end{lstlisting}
Contoh di atas menunjukkan terdapat dua variabel \verb|mhs1| dan \verb|mhs2 |bertipe struct \verb|Mahasiswa|.

\subsection{Akses Member Struct}
Bagaimana cara untuk mengakses member dari variabel struct yang telah dibuat? \\
Untuk mengakses member-member dari struct, digunakan operator dot (.) setelah nama variabelnya.
\begin{lstlisting}[language=c]
    <nama_variabel>.<member_struct>
\end{lstlisting}

Contoh:
\begin{lstlisting}[language=c]
    mhs1.age = 20;
    mhs1.nama = iqbal;
    
    mhs2.nama = fatur;
    mhs2.age = 21;
\end{lstlisting}

\subsection{Tugas Pendahuluan}
\begin{enumerate}
    \item Buatlah sebuah struct yang merepresentasikan informasi tentang seorang mahasiswa, yang memiliki nama, nim, dan nilai IPK. Kemudian, buatlah program untuk menginput data mahasiswa, menampilkan data mahasiswa, dan menghitung rata-rata IPK dari sejumlah mahasiswa.
    \item Anda diberikan struct yang merepresentasikan titik dalam sistem koordinat dua dimensi (x, y). Buatlah sebuah program C untuk menghitung jarak antara dua titik yang diinputkan oleh pengguna menggunakan rumus jarak Euclidean.
\end{enumerate}

\section{Algoritma Sorting}
Sorting merupakan suatu proses penyortiran atau pengurutan sebuah data.\\
Terdapat 2 macam pengurutan data pada sorting yaitu :
\begin{enumerate}
    \item Berdasarkan ascending (kecil ke besar).
    \item Berdasarkan Descending (besar ke kecil).
\end{enumerate}

\subsection{Bubble Sort}
Bubble sort merupakan algoritma pengurutan yang membandingkan dua data yang berdekatan dan menukarnya sampai tidak dalam urutan yang diinginkan.
Bubble sort menggunakan teknik iterasi. Iterasi merupakan proses melakukan perulangan sebanyak data yang diketahui.
Intinya pada iterasi melakukan perbandingan antara dua data.

\begin{figure}[H]
    \centering
    \includegraphics[width=0.7\linewidth]{P4/img/screenshot006.png}
    \caption{}
    \label{fig:enam}
\end{figure}

\begin{lstlisting}[language=c,caption=Implementasi Bubble Sort]
void swap ( int * xp , int * yp ) {
   int temp = *xp;
   *xp = *yp;
   *yp = temp;
}

void bubbleSort(int arr[], int n) {
   int i, j, swapped;        // dioptimasi dengan bool `swapped`:
   for (i = 0; i < n-1; i++) {
      swapped = 0;
      for (j = 0; j < n-i-1; j++) {
         if (arr[j] > arr[j+1]) {
            swap(&arr[j], &arr[j+1]);
            swapped = 1;
         }
      }
      if (swapped == 0)
         break;
   }
}
\end{lstlisting}

\subsection{Insertion Sort}
Insertion sort merupakan teknik sorting dengan cara menyisipkan atau memasukan setiap elemen secara berulang berulang.
Konsep insertion sort bisa diibaratkan sebuah kartu.

\begin{figure}[H]
    \centering
    \includegraphics[width=0.5\linewidth]{P4/img/screenshot007.png}
    \caption{}
    \label{fig:tujuh}
\end{figure}
\begin{lstlisting}[language=c,caption=Implementasi Insertion Sort], 
void insertionSort(int arr[]. int n) {
   int i, key, j;
   for (i = 1; i < n; i++) {
      key = arr[i];
      j = i-1;
    
      while (j >= 0 && arr[j] > key) {
         arr[j+1] = arr[j];
         j = j-1;
      }
      arr[j+1] = key;
   }
}
\end{lstlisting}

\begin{center}
    \colorbox{pink}{\parbox{0.8\linewidth}{\textbf{Catatan:} Terdapat berbagai algoritma sorting lain. Pelajari secara mandiri}}
\end{center}

\subsection{Tugas Pendahuluan}
\begin{enumerate}
    \item Urutkan array berikut menggunakan algoritma Bubble Sort:

          Array: [5, 2, 9, 1, 5, 6]
    \item Hitung kompleksitas waktu (Big O) dari algoritma Insertion Sort saat mengurutkan sebuah array dengan panjang n, dan jelaskan bagaimana kompleksitas ini dihitung.
\end{enumerate}

\section{Algoritma Searching}
Searching merupakan proses pencarian sebuah data yang diinginkan.

\subsection{Linear Search}
Linear Search bekerja dengan melakukan pengecekan kepada semua elemen yang ada.\\
Secara garis besar, cara kerja Linear Search adalah:

\begin{enumerate}
    \item Memeriksa item satu per satu.
    \item Apabila ditemukan, maka “ketemu”.
    \item Jika sampai akhir belum ditemukan, maka item yang dicari tidak ada.
\end{enumerate}

\begin{lstlisting}[language=c,caption=Implementasi Linear Search], 
int linearSearch(int arr[], int n, int item) {
    int i;
    for(i = 0; i < n; ++i) {
        if(item == arr[i])
          return 1;
    }
    return -1;
}
\end{lstlisting}

\subsection{Binary Search}
Binary Search adalah teknik pencarian di mana untuk setiap iterasinya kita membagi space pencarian menjadi hanya setengah
dari space pencarian awal hingga kita menemukan yang kita cari.

\begin{lstlisting}[language=c,caption=Implementasi Binary Search],   
bool f(int k, int a, int b, int n) {
   return ((k/a) * (b/a) >= n);
}

int binser(int a, int b, int n) {
   int l = 1;
   int r = 100000;
   while (r - l > 1) {
      int mid = (l + r) >> 1;
      bool can = f(mid);
      if(can)
         r = mid;
      else
         l = mid + 1;
   }
   if (can(l))
      return l;
   else
      return r;
}
\end{lstlisting}

\begin{center}
    \colorbox{pink}{\parbox{0.8\linewidth}{\textbf{Catatan:} Terdapat berbagai algoritma searching lain. Pelajari secara mandiri}}
\end{center}

\subsection{Tugas Pendahuluan}
\begin{enumerate}
    \item Anda memiliki daftar nama berikut: ["Alice", "Bob", "Charlie", "David", "Eve", "Frank"].
          Gunakan algoritma binary search untuk mencari apakah nama "Eve" ada dalam daftar ini.
          Jika ya, berapa langkah yang dibutuhkan?
    \item Jelaskan perbedaan antara pencarian linear (sequential search) dan pencarian biner (binary search).
          Kapan Anda akan memilih salah satu metode ini daripada yang lain?
\end{enumerate}

\end{document}